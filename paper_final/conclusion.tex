\section{Conclusion}\label{sec:conclusion}

%In the paper, we studied the constraints switch architectures apply on
%algorithm implementations in order to maintain high packet processing.
%Focusing on heavy hitter detection, we design an algorithm that follows these
%restrictions while still achieving high accuracy.

In this paper, we proposed an algorithm to detect heavy traffic flows within the
constraints of emerging programmable switches, and making this information
available within the switch itself, as packets are processed. Our solution,
\TheSystem, uses a pipeline of hash tables to track heavy flows preferentially,
by evicting lighter flows from switch memory over time. We prototype \TheSystem
with P4, walking through the switch programming features used to implement our
algorithm. Through simulations on a real traffic trace, we showed that
\TheSystem achieves high accuracy in finding heavy flows within the memory
constraints of switches today.%%  while outperforming existing heavy-hitter
%% detection schemes that use sampling and sketching.
%% In this paper, we discussed the constraints imposed by switching architectures on
%% algorithmic implemenations due to the need for high speed packet processing. In
%% particular, we propose an algorithm for heavy hitter detection called \TheSystem
%% that respects these restrictions while still achieving high accuracy. \TheSystem
%% uses a pipeline of hash tables along with a local minimum computation at each
%% pipeline stage to preferentially track the heavier flows. We also prototype the
%% algorithm in P4. Through our simulations, we show that \TheSystem achieves lower
%% error rates than existing sampling and sketching based approaches %like Sample
%%                                 %and Hold and Count-Min Sketch 
%% at comparable amounts of memory.
%% We are further studying \TheSystem through synthetic workloads, additional
%% optimizations to the algorithm, and theoretical analyses.

%univmon, dataplane
% assymmetric memory split
%synthetic workloads
%compiling to a target
%extensive analysis
